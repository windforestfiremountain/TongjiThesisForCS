\chapter{第三章标题}
\label{chap:03title}

% [本章引言]
% 写作策略:
% 1. 承上启下:简述上一章(如果有)或研究背景,指出当前问题。
% 2. 核心痛点:
% 3. 本章概述:。
% 突出创新点和标题的联系等等

\section{引言}
\label{sec:intro}

\subsection{研究背景与动机}
% 写作策略:展开描述选题的背景和动机
\subsection{本章主要贡献}
% 写作策略:列点陈述,清晰明了。
本章的主要贡献如下:
\begin{itemize}
    \item 我是贡献1
    \item 我是贡献2
    \item 我是贡献3
\end{itemize}

\section{工作内容1}
\label{sec:work_content1}
% 核心工作1:你做的事情



\subsection{问题定义}
% 给问题下定义,是什么任务,总述性去叙述



\subsection{模型整体架构}
% 写作策略:重点描述Conformer Backbone。
% 画图占位符:

\subsection{损失函数与优化目标}
% 写作策略:解释为什么用CTC(解决输入输出长度不一致问题)。
% 给出CTC公式。
由于输入的内容表征帧数与目标音素序列的长度不一致,本研究采用连接时序分类(CTC)损失函数来自动学习对齐路径......

\section{实验结果与分析}
\label{sec:experiments}

\subsection{实验设置}
% 写作策略:描述数据划分和训练细节。

\subsection{评价指标}
% 写作策略:描述数据划分和训练细节。

\subsection{不同骨干网络的性能对比}
% 写作策略:对比实验。


从表中可以看出,我们的模型取得了最优性能...明......

\subsection{消融实验与超参数分析}
% 写作策略:探究超参数影响。
% 内容:学习率 (Learning Rate)、学习率调度器 (Scheduler) 的影响。

\subsection{案例分析}
% 写作策略:探究超参数影响。

\section{本章小结}
\label{sec:summary}

