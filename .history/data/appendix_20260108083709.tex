% 附录, 注意多个附录的编号是按照字母排序的
% 此处补充计算机领域常见的Prompt模板附录,以figure的形式展示
\chapter{大模型打分Prompt模板}
\begin{figure}[H] % [H] 强制紧跟文字
    \centering
    % 调整1: fontupper改回 \small,保证清晰度
    % 调整2: boxsep改回 3pt,增加内部留白,看起来不拥挤
    \begin{tcolorbox}[colback=white, colframe=black, boxrule=0.8pt, arc=0pt, outer arc=0pt, 
        boxsep=3pt, left=4pt, right=4pt, top=4pt, bottom=4pt, 
        fontupper=\small] 
        
        \textbf{[System Prompt / 角色设定]} \\
        请定义大模型的角色,并撰写相应的任务。
        
        \textbf{任务说明:}
        % 列表保持紧凑,但行间距留一点点空隙 (itemsep=1pt)
        \begin{itemize}[noitemsep, topsep=2pt, parsep=0pt, itemsep=1pt, leftmargin=*]
            \item \textbf{输入数据:} 这是一组输入数据
            \item \textbf{核心逻辑:} 这是一组核心逻辑。
            \item \textbf{评分标准:} 这是一组评分标准。
            \item \textbf{输出格式:} 这是一组输出格式。
        \end{itemize}

        \vspace{0.2cm} 
        \hrule
        \vspace{0.2cm}

        \textbf{[User Input / 用户指令]} \\
        这是一组用户指令。以csv输入为例,包含输入数据、核心逻辑、评分标准和输出格式。
        
        \vspace{0.15cm}
        \textit{Input CSV Snippet:}
        
        % 调整3: 表格稍微拉开一点,看起来更像数据
        {\ttfamily \small
        \begin{tabular}{@{}llll@{}}
        id, & Column1, & Column2 & ... \\
        1, & xxx,    & xxx & ... \\
        2, & xxx,    & xxx & ... \\
        3, & xxx,    & xxx & ... \\
        \end{tabular}
        }
        
        \vspace{0.15cm}
        \textbf{示例 (Few-shot):} 
        \begin{itemize}[noitemsep, topsep=2pt, leftmargin=*]
            \item 输入: 这是一组输入数据
            \item 输出: 这是一组输出数据
        \end{itemize}

    \end{tcolorbox}
    
    \vspace{-0.2cm} % 如果标题离得太远,稍微拉近一点
    \caption{大模型打分Prompt模板}
    \label{fig:prompt_design}
\end{figure}

\chapter{外文资料原文}
\label{chap:appx:1}
As one of the most widely used techniques in operations research, {\em mathematical programming} is defined as a means of maximizing a quantity known as {\em objective function}, subject to a set of constraints represented by equations and inequalities.
Some known subtopics of mathematical programming are linear programming, nonlinear programming, multiobjective programming, goal programming, dynamic programming, and multilevel programming$^{[1]}$.


\begin{equation} \equcaption{Optimization Problem 1} % 指定其出现在索引中
    \left\{ \begin{array}{l}
        \max \,\,f(x)      \\ [0.1 cm]
        \mbox{subject to:} \\ [0.1 cm]
        \qquad g_j(x)\le 0, \quad j=1,2,\cdots,p
    \end{array} \right.
\end{equation}

\begin{equation}
    \left\{ \begin{array}{l}
        \max \,\,f\dual(x) \\ [0.1 cm]
        \mbox{subject to:} \\ [0.1 cm]
        \qquad g_j\dual(x)\ge 0, \quad j=1,2,\cdots,p
    \end{array} \right.
\end{equation}

\chapter{其它附录}
\label{chap:appx:2}
其它附录的内容可以放到这里。
也可以独立存放,然后将 \verb|\tongjiinput|(或者 \verb|\wyqyinput|)到主文件中。
